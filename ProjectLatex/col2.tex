
         %8
	\item Область задана на плоскости системой неравенств
        \[\left\{
	\begin{gathered}
	x^2+y^2 \geq 4 \\
	y^2 \leq 4\qquad\\\
	y \geq |x|\qquad
	\end{gathered}
	\right.\]\\
        Найдите её площадь. \\ [0.2cm]
	\begin{tabular}{*{4}{p{0.13\textwidth}}p{0.34\textwidth}}
	$A)\;4-\pi$ & $B)\;8-\pi$ & $C)\;4-2\pi$ & $D)\;8-2\pi$ & $E)\;$ нет правильного ответа
	\end{tabular}\\
        %9
	\item Найдите область определения функции \[y = \log_{\frac{1}{2}}{(\dfrac{-\sqrt{-x+7}}{-2x^2+17x-8})}\]\\
	\begin{tabular}{*{4}{p{0.13\textwidth}}p{0.34\textwidth}}
		$A)\;(\dfrac{1}{2}; 7)$ & $B)\;(8; +\infty)$ & $C)\;(-\infty; 7]$ & $D)\;(-\infty;\dfrac{1}{2})$ & $E)\;$ нет правильного ответа
	\end{tabular}\\
        %10
	\item Найдите наибольшее и наименьшее значения функции $f(x) = \sin^2(x) + \cos(2x) - \dfrac{1}{2}$. \\ [0.2cm]
	\begin{tabular}{*{4}{p{0.13\textwidth}}p{0.34\textwidth}}
		$A)\;\dfrac{3}{2}$ и $-\dfrac{3}{4}$ & $B)\;\dfrac{1}{2}$ и $-\dfrac{1}{2}$ & $C)\;\dfrac{3}{2}$ и $-\dfrac{3}{2}$ & $D)\;\dfrac{3}{4}$ и $-\dfrac{1}{2}$ & $E)\;$ нет правильного ответа
	\end{tabular}\\
        %11
	\item Найдите $\cos{(\dfrac{5\pi}{2} + 2\alpha)}$, если $\cos{\alpha} = \dfrac{3}{5}$, $-\dfrac{\pi}{2} < \alpha < 0$.\\ [0.2cm]
	\begin{tabular}{*{4}{p{0.13\textwidth}}p{0.34\textwidth}}
		$A)\;-\dfrac{24}{25}$ & $B)\;-\dfrac{7}{25}$ & $C)\;\dfrac{7}{25}$ & $D)\;\dfrac{24}{25}$ & $E)\;$ нет правильного ответа
	\end{tabular}\\
        %12
	\item Прямая $l$ задана на плоскости уравнением $3y-4x-5=0$. Укажите уравнение прямой, перпендикулярной прямой $l$ и проходящей через точку $A(2; -1)$.\\ [0.2cm]
	\begin{tabular}{*{3}{p{0.35\textwidth}}}
		$A)\;4y-3x+10=0$ & $B)\;4y+3x-2=0$ & $C)\;-\dfrac{1}{3}y+\dfrac{1}{4}x-\dfrac{5}{6}=0$ \\ [0.1cm] $D)\;-\dfrac{1}{3}y+\dfrac{1}{4}x+\dfrac{1}{5}=0$ & $E)\;$нет правильного ответа &
	\end{tabular}\\
        %13
	\item График функции $f(x) = 3x^2-18x+32$ сдвинули на 2 единицы влево и на 3 единицы вниз, получив при этом график функции $g(x)$. Какой вид может иметь $g(x)$? \\ [0.2cm]
	\begin{tabular}{*{3}{p{0.35\textwidth}}}
		$A)\;3x^2-10x+77$ & $B)\;3x^2-6x+5$ & $C)\;3x^2-10x+83$ \\ [0.1cm] $D)\;3x^2-6x+11$ & $E)\;$нет правильного ответа &
	\end{tabular}\\
 %14
	\item $a_n$ — возрастающая арифметическая прогрессия, причем $a_1>0$. Найдите $a_4$, если $a_2a_6=105$ и $a_3a_5=117$. \\ [0.2cm]
	\begin{tabular}{*{4}{p{0.13\textwidth}}p{0.34\textwidth}}
		$A)\;7$ & $B)\;11$ & $C)\;13$ & $D)\;17$ & $E)\;$ нет правильного ответа
	\end{tabular}\\
        %15
	\item Укажите вариант ответа, в котором перечислены все верные утверждения:
        \begin{itemize}
		\item [\((a)\)] Для любого треугольника центр вписанной в него окружности совпадает с центром описанной вокруг него окружности;
		\item [\((b)\)] Диагонали ромба перпендикулярны и точкой пересечения делятся пополам;
		\item [\((c)\)] Биссектрисы треугольника точкой пересечения делятся в отношении $2:1$, считая от вершины.
	\end{itemize}
	\begin{tabular}{*{4}{p{0.13\textwidth}}p{0.34\textwidth}}
		$A)\;a$ & $B)\;a$, $b$ & $C)\;b$, $c$ & $D)\;b$ & $E)\;$ нет правильного ответа
	\end{tabular}\\