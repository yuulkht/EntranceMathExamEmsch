
	%25
	\item Катя ехала от экономического факультета до пансионата <<Фейерверк>>, а Влад – наоборот. Они встретились, когда Катя проехала 10 км и еще четверть оставшегося ей до пансионата пути, а Влад проехал 20 км и треть оставшегося ему до экономического факультета пути. Какое расстояние между экономическим факультетом и пансионатом <<Фейерверк>>?\\[0.2cm]
	\begin{tabular}{*{4}{p{0.13\textwidth}}p{0.34\textwidth}}
		$A)\;50$ км & $B)\;60$ км & $C)\;70$ км& $D)\;80$ км & $E)\;$ нет правильного ответа
	\end{tabular}\\
	%26
	\item Школьники, обладающие одинаковой производительностью решения задач, собрались на досуге решить 360 задач по геометрии (каждую задачу решает 1 человек 1 раз). К сожалению, один из школьников приболел, поэтому вместо него задачи отправился решать студент Саша, производительность решения задач которого в три раза больше производительности каждого
из школьников. Поэтому каждый школьник в действительности решил на 6 задач меньше, чем планировалось. Все школьники и Саша решали задачи одинаковое время. Сколько школьников в действительности решало задачи?\\ [0.2cm]
	\begin{tabular}{*{4}{p{0.13\textwidth}}p{0.34\textwidth}}
		$A)\;6$ & $B)\;9$ & $C)\;10$ & $D)\;12$ & $E)\;$нет правильного ответа
	\end{tabular}\\
	%27
	\item На вступительных тестах ЭМШ все школьники в аудитории сели так, что за каждой партой их оказалось по двое. Парт, за которыми сидят две девочки, втрое больше, чем парт, за которыми сидят мальчик с девочкой. А парт, за которыми сидят двое мальчиков, вдвое больше, чем парт, за которыми сидят две девочки. Сколько в аудитории мальчиков, если известно, что девочек там 14?\\[0.2cm]
	\begin{tabular}{*{4}{p{0.13\textwidth}}p{0.34\textwidth}}
		$A)\;20$ & $B)\;22$ & $C)\;24$ & $D)\;26$ & $E)\;$нет правильного ответа
	\end{tabular}\\
	%28
	\item Преподаватели курса в ЭМШ решили проверить у школьников ДЗ, состоящее из кейса, кроссворда и эссе. Среди 43 школьников курса кейс решили 12, кроссворд — 15, а написали эссе — 12 человек. Кроме того, кейс и эссе сделали 3 человека, эссе и кроссворд — 5 человек, а кейс и кроссворд — 4 человека. Наконец, все 3 задания выполнил только 1 человек. Сколько на курсе школьников, которые вообще не сделали ДЗ?\\
 [0.2cm]
	\begin{tabular}{*{4}{p{0.13\textwidth}}p{0.34\textwidth}}
		$A)\;12$ & $B)\;13$ & $C)\;14$ & $D)\;15$ & $E)\;$нет правильного ответа
	\end{tabular}\\
 %29
	\item Кате очень понравилась лекция про чётность и нечётность на одном из курсов в ЭМШ. После пары она записала на доске несколько последовательных натуральных чисел и подсчитала количество четных и нечетных. Оказалось, что $48\%$ чисел на доске — нечетные. Сколько всего четных чисел записано на доске? \\[0.2cm]
	\begin{tabular}{*{4}{p{0.13\textwidth}}p{0.34\textwidth}}
		$A)\;12$ & $B)\;13$ & $C)\;14$ & $D)\;15$ & $E)\;$нет правильного ответа
	\end{tabular}\\
        %30
	\item Аня выяснила, что подарочных стикеров ЭМШ осталось 525, а книг ЭМШ — 735. Она решила составить из них одинаковые наборы, причем так, чтобы раздать их наибольшему количеству детей и использовать все книги и все стикеры. Сколько наборов сможет собрать Аня?\\ [0.2cm]
	\begin{tabular}{*{3}{p{0.13\textwidth}}p{0.15\textwidth}p{0.34\textwidth}}
		$A)\; 35$  & $B)\; 75$  & $C)\; 105$ & $D)\;125$ & $E)\;$нет правильного ответа
	\end{tabular}\\