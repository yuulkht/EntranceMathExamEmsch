        %16
	\item В трапеции $ABCD$ с основаниями $AD || BC$ диагонали пересекаются в точке $O$. Известны площади $S(ABCD) = 32$ и $S(\Delta BCO) = 2$. Найдите площадь треугольника $AOD$.\\ [0.2cm]
	\begin{tabular}{*{4}{p{0.13\textwidth}}p{0.34\textwidth}}
		$A)\;6$ & $B)\;8$ & $C)\;16$ & $D)\;18$ & $E)\;$ нет правильного ответа
	\end{tabular}\\
	%17
	\item Из точки $M$ провели прямую, касающуюся окружности в точке $A$. Перпендикуляно $AM$ провели секущую, проходящую через точку $M$. Оказалось, что $AM = 12$, а внутренняя часть секущей равна 10. Найдите радиус окружности.  \\ [0.2cm]
	\begin{tabular}{*{4}{p{0.13\textwidth}}p{0.34\textwidth}}
		$A)\;12$ & $B)\;13$ & $C)\;17$ & $D)\;2\sqrt{61}$ & $E)\;$ нет правильного ответа
	\end{tabular}\\	
	%18
	\item Отрезки $AM$ и $BH$ – соответственно медиана и высота остроугольного треугольника $ABC$. Известно, что  $AH = 1$  и  $2\angle MAC = \angle MCA$.  Найдите сторону BC.
	\\[0.2cm]
	\begin{tabular}{*{4}{p{0.13\textwidth}}p{0.34\textwidth}}
		$A)\;0,5$ & $B)\;1$ & $C)\;2$ & $D)\;3$ & $E)\;$ нет правильного ответа
	\end{tabular}\\
        %19
	\item Найдите остаток от деления $2^{100000}$ на $31$.\\ [0.2cm]
	\begin{tabular}{*{4}{p{0.13\textwidth}}p{0.34\textwidth}}
		$A)\;1$ & $B)\;3$ & $C)\;13$ & $D)\;30$ & $E)\;$ нет правильного ответа
	\end{tabular}\\
        %20
	\item При каких значениях параметра $a$ корни уравнения $x^2+8x+2a=0$ существуют и все принадлежат отрезку $[-6, -3]$?\\ [0.2cm]
	\begin{tabular}{*{4}{p{0.13\textwidth}}p{0.34\textwidth}}
		$A)\;[6;$ $8)$ & $B)\;(7,5;$ $8]$ & $C)\;[6;$ $8]$ & $D)\;[7,5;$ $8]$ & $E)\;$ нет правильного ответа
	\end{tabular}\\
        %21
	\item Функция $f(x)$ определена для $x \geq 0$, причем для любых положительных $a$ и $b$ верно, что $f(ab) = f(a) + f(b)$. Найдите $f(2024)$, если $f(\dfrac{1}{2024}) = 1$. \\ [0.2cm]
	\begin{tabular}{*{4}{p{0.13\textwidth}}p{0.34\textwidth}}
	$A)\;-1$ & $B)\;\dfrac{1}{2024}$ & $C)\;2024$ & $D)\;1$ & $E)\;$ нет правильного ответа
	\end{tabular}\\
 	%22
	\item Решите уравнение \[||4-x^2|-x^2|=1\] \\
        \begin{tabular}{*{3}{p{0.35\textwidth}}}
		$A)\;$нет решений & $B)\;\{\pm \sqrt{\dfrac{3}{2}}\}$ & $C)\;\{\sqrt{\dfrac{3}{2}}; \sqrt{\dfrac{5}{2}}\}$ \\ $D)\;\{\pm \sqrt{\dfrac{3}{2}}; \pm \sqrt{\dfrac{5}{2}}\}$ & $E)\;$нет правильного ответа &
	\end{tabular}\\
 %23
	\item Братья Игорь и Костя привезли в чемоданах сладости на Выездную школу ЭМШ. Когда школа закончилась, оказалось, что  общий вес чемоданов братьев за время
школы уменьшился на 10\%. При этом вес чемодана Игоря уменьшился на 15\%, а вес
чемодана Кости — на 6\%. Известно также, что в конце Выездной школы чемодан Кости весил на $7$ кг больше, чем чемодан Игоря в начале школы. Определите первоначальный
вес чемоданов Игоря и Кости. В ответе укажите их сумму.  \\ [0.2cm]
	\begin{tabular}{*{4}{p{0.13\textwidth}}p{0.34\textwidth}}
	$A)\;60$ кг& $B)\;70$ кг& $C)\;80$ кг & $D)\;90$ кг & $E)\;$ нет правильного ответа
	\end{tabular}\\
	%24
	\item Александр вчетверо старше Николая. Сумма их возрастов — 80 лет. Через сколько лет Александр будет втрое старше Николая?\\[0.2cm]
	\begin{tabular}{*{4}{p{0.13\textwidth}}p{0.34\textwidth}}
		$A)\;6$ & $B)\;9$ & $C)\;12$ & $D)\;15$ & $E)\;$ нет правильного ответа
	\end{tabular}\\